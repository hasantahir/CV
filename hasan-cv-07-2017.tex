%%%%%%%%%%%%%%%%%%%%%%%%%%%%%%%%%%%%%%%%%%%%%%%%%%%%%%%%%%%%%%%%%%%%%%%%
%%%%%%%%%%%%%%%%%%%%%% Simple LaTeX CV Template %%%%%%%%%%%%%%%%%%%%%%%%
%%%%%%%%%%%%%%%%%%%%%%%%%%%%%%%%%%%%%%%%%%%%%%%%%%%%%%%%%%%%%%%%%%%%%%%%

%%%%%%%%%%%%%%%%%%%%%%%%%%%%%%%%%%%%%%%%%%%%%%%%%%%%%%%%%%%%%%%%%%%%%%%%
%% NOTE: If you find that it says                                     %%
%%                                                                    %%
%%                           1 of ??                                  %%
%%                                                                    %%
%% at the bottom of your first page, this means that the AUX file     %%
%% was not available when you ran LaTeX on this source. Simply RERUN  %%
%% LaTeX to get the ``??'' replaced with the number of the last page  %%
%% of the document. The AUX file will be generated on the first run   %%
%% of LaTeX and used on the second run to fill in all of the          %%
%% references.                                                        %%
%%%%%%%%%%%%%%%%%%%%%%%%%%%%%%%%%%%%%%%%%%%%%%%%%%%%%%%%%%%%%%%%%%%%%%%%

%%%%%%%%%%%%%%%%%%%%%%%%%%%% Document Setup %%%%%%%%%%%%%%%%%%%%%%%%%%%%

% Don't like 10pt? Try 11pt or 12pt
\documentclass[10pt]{article}
\RequirePackage[T1]{fontenc}

% LaTeX will typeset using Computer Modern Roman, which a lot of
% non-mathematicians and non-engineers won't like. Also, a few PDF
% viewers may not render CMR very well. Instead, Times New Roman can
% be used. That's what this package does.
\usepackage{mathptmx}

% The automated optical recognition software used to digitize resume
% information works best with fonts that do not have serifs. This
% command uses a sans serif font throughout. Uncomment both lines (or at
% least the second) to restore a Roman font (i.e., a font with serifs).
% (NOTE: This requires the times package above)
%\renewcommand{\familydefault}{\sfdefault}

% This is a helpful package that puts math inside length specifications
\usepackage{calc}

% This package helps LaTeX auto-hyphenate hyphenated words if you use
% special hyphens. For example, bio\-/mimicry will properly hyphenate
% ``mimicry'' if necessary.
\usepackage[shortcuts]{extdash}

% Layout: Puts the section titles on left side of page
\reversemarginpar

%
%         PAPER SIZE, PAGE NUMBER, AND DOCUMENT LAYOUT NOTES:
%
% The next \usepackage line changes the layout for CV style section
% headings as marginal notes. It also sets up the paper size as either
% letter or A4. By default, letter was used. If A4 paper is desired,
% comment out the letterpaper lines and uncomment the a4paper lines.
%
% As you can see, the margin widths and section title widths can be
% easily adjusted.
%
% ALSO: Notice that the includefoot option can be commented OUT in order
% to put the PAGE NUMBER *IN* the bottom margin. This will make the
% effective text area larger.
%
% IF YOU WISH TO REMOVE THE ``of LASTPAGE'' next to each page number,
% see the note about the +LP and -LP lines below. Comment out the +LP
% and uncomment the -LP.
%
% IF YOU WISH TO REMOVE PAGE NUMBERS, be sure that the includefoot line
% is uncommented and ALSO uncomment the \pagestyle{empty} a few lines
% below.
%

%% Use these lines for letter-sized paper
\usepackage[paper=letterpaper,
            %includefoot, % Uncomment to put page number above margin
            marginparwidth=1.2in,     % Length of section titles
            marginparsep=.05in,       % Space between titles and text
            margin=1in,               % 1 inch margins
            includemp]{geometry}

%% Use these lines for A4-sized paper
%\usepackage[paper=a4paper,
%            %includefoot, % Uncomment to put page number above margin
%            marginparwidth=30.5mm,    % Length of section titles
%            marginparsep=1.5mm,       % Space between titles and text
%            margin=25mm,              % 25mm margins
%            includemp]{geometry}

%% More layout: Get rid of indenting throughout entire document
\setlength{\parindent}{0in}

% Provides special list environments and macros to create new ones
\usepackage[shortlabels]{enumitem}

% Simpler bibsections for CV sections
% (thanks to natbib for inspiration)
%
% * For lists of references with hanging indents and no numbers:
%
%   \begin{bibsection}
%       \item ...
%   \end{bibsection}
%
% * For numbered lists of references (with hanging indents):
%
%   \begin{bibenum}
%       \item ...
%   \end{bibenum}
%
%   Note that bibenum numbers continuously throughout. To reset the
%   counter, use
%
%   \restartlist{bibenum}
%
%   at the place where you want the numbering to reset.

\makeatletter
\newlength{\bibhang}
\setlength{\bibhang}{1em}
\newlength{\bibsep}
 {\@listi \global\bibsep\itemsep \global\advance\bibsep by\parsep}
\newlist{bibsection}{itemize}{3}
\setlist[bibsection]{label=,leftmargin=\bibhang,%
        itemindent=-\bibhang,
        itemsep=\bibsep,parsep=\z@,partopsep=0pt,
        topsep=0pt}
\newlist{bibenum}{enumerate}{3}
\setlist[bibenum]{label=[\arabic*],resume,leftmargin={\bibhang+\widthof{[999]}},%
        itemindent=-\bibhang,
        itemsep=\bibsep,parsep=\z@,partopsep=0pt,
        topsep=0pt}
\let\oldendbibenum\endbibenum
\def\endbibenum{\oldendbibenum\vspace{-.6\baselineskip}}
\let\oldendbibsection\endbibsection
\def\endbibsection{\oldendbibsection\vspace{-.6\baselineskip}}
\makeatother

%% Reference the last page in the page number
%
% NOTE: comment the +LP line and uncomment the -LP line to have page
%       numbers without the ``of ##'' last page reference)
%
% NOTE: uncomment the \pagestyle{empty} line to get rid of all page
%       numbers (make sure includefoot is commented out above)
%
\usepackage{fancyhdr,lastpage}
\pagestyle{fancy}
%\pagestyle{empty}      % Uncomment this to get rid of page numbers
\fancyhf{}\renewcommand{\headrulewidth}{0pt}
\fancyfootoffset{\marginparsep+\marginparwidth}
\newlength{\footpageshift}
\setlength{\footpageshift}
          {0.5\textwidth+0.5\marginparsep+0.5\marginparwidth-2in}
\lfoot{\hspace{\footpageshift}%
       \parbox{4in}{\, \hfill %
                    \arabic{page} of \protect\pageref*{LastPage} % +LP
%                    \arabic{page}                               % -LP
                    \hfill \,}}

% Finally, give us PDF bookmarks
\usepackage{color,hyperref}
\definecolor{darkblue}{rgb}{0.0,0.0,0.3}
\hypersetup{colorlinks,breaklinks,
            linkcolor=darkblue,urlcolor=darkblue,
            anchorcolor=darkblue,citecolor=darkblue}

%%%%%%%%%%%%%%%%%%%%%%%% End Document Setup %%%%%%%%%%%%%%%%%%%%%%%%%%%%


%%%%%%%%%%%%%%%%%%%%%%%%%%% Helper Commands %%%%%%%%%%%%%%%%%%%%%%%%%%%%

%%% HEADING AT TOP OF CURRICULUM VITAE

% The title (name) with a horizontal rule under it
% (optional argument typesets an object right-justified across from name
%  as well)
%
% Usage: \makeheading{name}
%        OR
%        \makeheading[right_object]{name}
%
% Place at top of document. It should be the first thing.
% If ``right_object'' is provided in the square-braced optional
% argument, it will be right justified on the same line as ``name'' at
% the top of the CV. For example:
%
%       \makeheading[\emph{Curriculum vitae}]{Your Name}
%
% will put an emphasized ``Curriculum vitae'' at the top of the document
% as a title. Likewise, a picture could be included:
%
%   \makeheading[{\includegraphics[height=1.5in]{my_picture}}]{Your Name}
%
% the picture will be flush right across from the name. For this example
% to work, make sure the extra set of curly braces is included. Also
% makes ure that \usepackage{graphicx} is somewhere in the preamble.
\newcommand{\makeheading}[2][]%
        {\hspace*{-\marginparsep minus \marginparwidth}%
         \begin{minipage}[t]{\textwidth+\marginparwidth+\marginparsep}%
             {\large \bfseries #2 \hfill #1}\\[-0.15\baselineskip]%
                 \rule{\columnwidth}{1pt}%
         \end{minipage}}

%%% SECTION HEADINGS

% The section headings. Flush left in small caps down pseudo-margin.
%
% Usage: \section{section name}
\renewcommand{\section}[1]{\pagebreak[3]%
    \vspace{1.3\baselineskip}%
    \phantomsection\addcontentsline{toc}{section}{#1}%
    \noindent\llap{\scshape\smash{\parbox[t]{\marginparwidth}{\hyphenpenalty=10000\raggedright #1}}}%
    \vspace{-\baselineskip}\par}

%%% LISTS

% This macro alters a list by removing some of the space that follows the list
% (is used by lists below)
\newcommand*\fixendlist[1]{%
    \expandafter\let\csname preFixEndListend#1\expandafter\endcsname\csname end#1\endcsname
    \expandafter\def\csname end#1\endcsname{\csname preFixEndListend#1\endcsname\vspace{-0.6\baselineskip}}}

% These macros help ensure that items in outer-type lists do not get
% separated from the next line by a page break
% (they are used by lists below)
\let\originalItem\item
\newcommand*\fixouterlist[1]{%
    \expandafter\let\csname preFixOuterList#1\expandafter\endcsname\csname #1\endcsname
    \expandafter\def\csname #1\endcsname{\let\oldItem\item\def\item{\pagebreak[2]\oldItem}\csname preFixOuterList#1\endcsname}
    \expandafter\let\csname preFixOuterListend#1\expandafter\endcsname\csname end#1\endcsname
    \expandafter\def\csname end#1\endcsname{\let\item\oldItem\csname preFixOuterListend#1\endcsname}}
\newcommand*\fixinnerlist[1]{%
    \expandafter\let\csname preFixInnerList#1\expandafter\endcsname\csname #1\endcsname
    \expandafter\def\csname #1\endcsname{\let\oldItem\item\let\item\originalItem\csname preFixInnerList#1\endcsname}
    \expandafter\let\csname preFixInnerListend#1\expandafter\endcsname\csname end#1\endcsname
    \expandafter\def\csname end#1\endcsname{\csname preFixInnerListend#1\endcsname\let\item\oldItem}}

% An itemize-style list with lots of space between items
%
% Usage:
%   \begin{outerlist}
%       \item ...    % (or \item[] for no bullet)
%   \end{outerlist}
\newlist{outerlist}{itemize}{3}
    \setlist[outerlist]{label=\enskip\textbullet,leftmargin=*}
    \fixendlist{outerlist}
    \fixouterlist{outerlist}

% An environment IDENTICAL to outerlist that has better pre-list spacing
% when used as the first thing in a \section
%
% Usage:
%   \begin{lonelist}
%       \item ...    % (or \item[] for no bullet)
%   \end{lonelist}
\newlist{lonelist}{itemize}{3}
    \setlist[lonelist]{label=\enskip\textbullet,leftmargin=*,partopsep=0pt,topsep=0pt}
    \fixendlist{lonelist}
    \fixouterlist{lonelist}

% An itemize-style list with little space between items
%
% Usage:
%   \begin{innerlist}
%       \item ...    % (or \item[] for no bullet)
%   \end{innerlist}
\newlist{innerlist}{itemize}{3}
    \setlist[innerlist]{label=\enskip\textbullet,leftmargin=*,parsep=0pt,itemsep=0pt,topsep=0pt,partopsep=0pt}
    \fixinnerlist{innerlist}

% An environment IDENTICAL to innerlist that has better pre-list spacing
% when used as the first thing in a \section
%
% Usage:
%   \begin{loneinnerlist}
%       \item ...    % (or \item[] for no bullet)
%   \end{loneinnerlist}
\newlist{loneinnerlist}{itemize}{3}
    \setlist[loneinnerlist]{label=\enskip\textbullet,leftmargin=*,parsep=0pt,itemsep=0pt,topsep=0pt,partopsep=0pt}
    \fixendlist{loneinnerlist}
    \fixinnerlist{loneinnerlist}

%%% EXTRA SPACE

% To add some paragraph space between lines.
% This also tells LaTeX to preferably break a page on one of these gaps
% if there is a needed pagebreak nearby.
\newcommand{\blankline}{\quad\pagebreak[3]}
\newcommand{\halfblankline}{\quad\vspace{-0.5\baselineskip}\pagebreak[3]}

%%% FORMATTING MACROS

% Provides a linked \doi{#1} that links doi:#1 to http://dx.doi.org/#1
\usepackage{doi}
% To change the text before the DOI, adjust this command
%\renewcommand\doitext{doi:}

% Provides a linked \url{#1} that doesn't require escape characters
\usepackage{url}

% You can adjust the style \url{} uses here:
% (options are: same, rm, sf, tt; defaults to tt)
\urlstyle{same}

% For \email{ADDRESS}, links ADDRESS to the url mailto:ADDRESS
% (uncomment to typeset the e\-/mail address in typewriter font;
%  otherwise, will be typeset in the \urlstyle above)
%\DeclareUrlCommand\emaillink{\urlstyle{tt}}
\providecommand*\emaillink[1]{\nolinkurl{#1}}
\providecommand*\email[1]{\href{mailto:#1}{\emaillink{#1}}}

\providecommand\BibTeX{{B\kern-.05em{\sc i\kern-.025em b}\kern-.08em \TeX}}
\providecommand\Matlab{\textsc{Matlab}}

% Custom hyphenation rules for words that LaTeX has trouble with
\hyphenation{bio-mim-ic-ry bio-in-spi-ra-tion re-us-a-ble pro-vid-er Media-Wiki}

%%%%%%%%%%%%%%%%%%%%%%%% End Helper Commands %%%%%%%%%%%%%%%%%%%%%%%%%%%

%%%%%%%%%%%%%%%%%%%%%%%%% Begin CV Document %%%%%%%%%%%%%%%%%%%%%%%%%%%%

\begin{document}
\makeheading{Hasan ~T.~Abbas}

\section{Contact Information}

% NOTE: Mind where the & separators and \\ breaks are in the following
%       table. Table is one row made up of three parboxes. The left
%       parbox has address info, the middle parbox has a vertical bar,
%       and the right parbox has phone and electronic contact
%       information.
%
% MACROS: \rcollength is the width of the right column of the table
%             (adjust it to your liking; default is 1.85in).
%         \spacewidth is width of area between left and right boxes.
%
\newlength{\rcollength}\setlength{\rcollength}{1.85in}%
\newlength{\spacewidth}\setlength{\spacewidth}{20pt}
%
\begin{tabular}[t]{@{}p{\textwidth-\rcollength-\spacewidth}@{}p{\spacewidth}@{}p{\rcollength}}%

% Address box
\parbox{\textwidth-\rcollength-\spacewidth}{%
PhD Electrical Engineering\\
\href{http://www.tamu.edu/}{Texas A\& M University}\\
\href{http://engineering.tamu.edu/electrical}{Department of Electrical \& Computer Engineering}\\
212-H WEB Texas A\&M University\\
College Station, TX 77843-3128, USA }


% Uncomment to add a vertical bar in middle of contact information
%{\vrule width 0.5pt}
\parbox[m][5\baselineskip]{\spacewidth}{} &

% Non-snail-mail contact information
\parbox{\rcollength}{%
\textit{Cell:} +1-979-422-5347 \\
\textit{Fax:} +1-979-845-6259 \\
\textit{E-mail:} \email{hasantahir@tamu.edu}\\
\textit{Website:} \href{http://www.hasantahir.github.io}{hasantahir.github.io}
}

\end{tabular}

%%
%% In modern CV's, it seems like ``Objective'' is frowned upon. Instead,
%% incorporate it into a well-constructed cover letter. The ``More
%% information'' can go at the end of the CV, but it should not distract
%% from the section giving references available to contact.
%%
%

%%%%%%%%%%%%%%%%%%%%%%%%%%
%%%%%%%%%%%%%%%%%%%%%%%%%% OBJECTIVE
%%%%%%%%%%%%%%%%%%%%%%%%%%

\section{Objective}

Placement in an academic position (i.e., faculty, postdoctoral, or
research scientist) that allows for advanced research in terahertz plasmonics (i.e., modeling, analysis, design, and verification) with a particular focus on the thin-layered semiconductor and graphene structures.
% \begin{innerlist}
% \item More information and auxiliary documents can be found at\\\url{http://www.tedpavlic.com/facjobsearch/}
% \end{innerlist}


%%%%%%%%%%%%%%%%%%%%%%%%%%
%%%%%%%%%%%%%%%%%%%%%%%%%% EDUCATION
%%%%%%%%%%%%%%%%%%%%%%%%%%

\section{Education}

\href{http://www.tamu.edu/}\textbf{{Texas A\& M University}},
College Station, TX
\begin{outerlist}

\item[] Ph.D.,
        \href{http://engineering.tamu.edu/electrical/}
             {Electrical and Computer Engineering},
             August 2012 - August 2017
        \begin{innerlist}

        \item Dissertation: \emph{Plasmonic Devices in the Terahertz and Optical Frequency domains}
        \item Adviser:
              \href{http://engineering.tamu.edu/electrical/people/rnevels}
                   {Professor Robert D. Nevels}
        \item GPA 4.0
        \item Area of Study: Numerical Electromagnetics and Plasmonics
        \end{innerlist}
\end{outerlist}
\halfblankline
\halfblankline

\href{http://uet.edu.pk/}\textbf{{University of Engineering \& Technology}},
Lahore, Pakistan
\begin{outerlist}
\item[] B.Sc.,
        \href{http://uet.edu.pk/faculties/facultiesinfo/department?RID=introduction&id=9}
             {Electrical Engineering}, July 2009
        \begin{innerlist}
        \item With Honors (absolute marks 76.4\%)
        \item Electrical specialization (emphasis in Telecommunication and Computer Science)
        \end{innerlist}

\end{outerlist}


%%%%%%%%%%%%%%%%%%%%%%%%%%
%%%%%%%%%%%%%%%%%%%%%%%%%% RESEARCH INTERESTS
%%%%%%%%%%%%%%%%%%%%%%%%%%


\section{Research Interests}

\textbf{Electromagnetics,
Plasmonics, Numerical Electromagnetics, Miniaturized on-chip Antennas,
Nanophotonics, Two-dimensional Physics and materials}



%%%%%%%%%%%%%%%%%%%%%%%%%%
%%%%%%%%%%%%%%%%%%%%%%%%%% WORK EXPERIENCE
%%%%%%%%%%%%%%%%%%%%%%%%%%


\section{Current Academic Appointments}

\textbf{Fulbright Scholar},
            \href{http://www.tamu.edu/}{Texas A\& M University}
            \hfill {August 2012 to August 2017}
\begin{innerlist}

    \item[] \href{http://engineering.tamu.edu/electrical}{Department of Electrical \& Computer Engineering}
    \begin{innerlist}
        \item Affiliations:
            \begin{innerlist}
                \item \href{http://ee.tamu.edu/~eml/}{Electromagnetics and Microwave Laboratory}
                \item \href{http://iqse.tamu.edu/}{Institute of Quantum Science and Engineering}
            \end{innerlist}
    \end{innerlist}

\end{innerlist}

\halfblankline

\textbf{Instructor},
            \href{http://www.tamu.edu/}{Texas A\& M University}
            \hfill {August 2014 to present}
\begin{innerlist}
        \item Courses:
            \begin{innerlist}
                \item \href{http://ee.tamu.edu/~eml/}{Electric and Magnetic Fields}
                \item \href{http://iqse.tamu.edu/}{Applied Electromagnetic Theory}
            \end{innerlist}
\end{innerlist}

\section{Previous Academic Appointments}

\textbf{Lecturer},
            \href{http://www.uet.edu.pk/}{University of Engineering \& Technology,
                                        Lahore, Pakistan}
            \hfill {August 2009 to August 2012}
\begin{innerlist}

    \item[] \href{http://www.uet.edu.pk/faculties/facultiesinfo/department?RID=introduction&id=30}{Department of Electrical Engineering \& Technology}
    \begin{innerlist}
        \item Courses:
            \begin{innerlist}
                \item {Electromagnetic Theory}
                \item {Antennas and Wave Propagation}
                \item {Applied Electromagnetics}
            \end{innerlist}
        \item Laboratories:
            \begin{innerlist}
                \item {Microwave and Antennas Laboratory}
                \item {Communication Systems}
            \end{innerlist}
    \end{innerlist}

\end{innerlist}

\halfblankline


% \section{Submitted Journal Publications}
%
% % Add a little space to nudge next ``Ref'd Journal Publications'' marginpar
% % down to make room for tall ``Submitted Journal Publications''
% % marginpar. If there are enough submitted journal publications, this
% % space will not be needed (and should be removed).
% \vspace{0.1in}

%%%%%%%%%%%%%%%%%%%%%%%%%%
%%%%%%%%%%%%%%%%%%%%%%%%%% PUBLICATIONS
%%%%%%%%%%%%%%%%%%%%%%%%%%


\section{Book Chapters}

\begin{bibenum}
    \item[2015]Robert D. Nevels, Hasan Tahir Abbas, ``Optical Nanoantennas'', In Chapter in Handbook of Antenna Technologies, Springer Singapore, pp. 1-33, 2015.
\end{bibenum}

% Add a little space to nudge next ``Conference Publications'' marginpar
% down to make room for tall ``Submitted Conference Publications''
% marginpar. If there are enough submitted journal publications, this
% space will not be needed (and should be removed).
\vspace{0.1in}

\section{Journal Publications}

\item[2017] H.T., Abbas, X., Zeng, M., AlAmri, R.D., Nevels, M.S., Zubairy, ``Nanoscopy using a semiconductor heterostructure as the sample stage'', submitted in Optics Express, 2017.
\vspace{0.1in}

\section{Conference Publications}

\begin{bibenum}

\item[2017] H.T., Abbas, R.D., Nevels, ``An Integral Equation Scheme for Plasma based Thin Sheets'', In Antennas and Propagation \& USNC/URSI National Radio Science Meeting, 2017 IEEE International Symposium on.
\item[2017] H.T., Abbas, R.D., Nevels, K.A., Michalski , ``Plasma based Terahertz devices'', Wireless \& Microwave Circuits \& Systems, 2017 IEEE Texas Symposium on.
\item[2016] H.T., Abbas, R.D., Nevels, ``Plasma based integrated on-chip antenna'', In Antennas and Propagation \& USNC/URSI National Radio Science Meeting, 2016 IEEE International Symposium on, pp. 1645-1646, 2016.
\item[2015] J., Shin, H.T., Abbas, R.D., Nevels, ``A numerical method for the electromagnetic field time domain propagator equations'', In Antennas and Propagation \& USNC/URSI National Radio Science Meeting, 2015 IEEE International Symposium on, pp. 1480-1481, 2015.
\item[2015] J., Shin, H.T., Abbas, R.D., Nevels, ``A numerical method for the electromagnetic field time domain propagator equations'', In Antennas and Propagation \& USNC/URSI National Radio Science Meeting, 2015 IEEE International Symposium on, pp. 1480-1481, 2015.
\item[2015] H.T., Abbas, J., Shin, R.D., Nevels, ``Numerical techniques for evaluating electromagnetic field propagators'', In Computational Electromagnetics (ICCEM), 2015 IEEE International Conference on, pp. 22-23, 2015.
\item[2014] R.D., Nevels, K.A., Michalski, H.T., Abbas, ``Plasmonic and surface wave propagation in boundary layers in the microwave, THz, and optical regimes'', In Antenna Measurements \& Applications (CAMA), 2014 IEEE Conference on, pp. 1-3, 2014.

\end{bibenum}

\section{Conference Talks}

\begin{bibenum}

    \item R.D. Nevels, K.A. Michalski, and H.T. Abbas
       Complex Plane Interpretation of Nano-Aperture Excited Plasmon Waves.
        In: \emph{University of Electronic Science and Technology China (UESTC) National Summer School}, Chengdu, China, July, 2015.

    \item R.D. Nevels, and H.T. Abbas
        A decomposition and interpretation of plasma and plasmonic waves.
        In: \emph{Institute for Quantum Science and Engineering Workshop}, College Station, TX, January 13--14, 2015.

    \item R.D. Nevels, L. Kish, and H.T. Abbas
        Twisted Waves: Concept and Limitations.
        In: \emph{2013 IEEE AP-S/USNC-URSI Symposium}, Orlando, FL, July 7--13, 2013.

\end{bibenum}



%%%%%%%%%%%%%%%%%%%%%%%%%%
%%%%%%%%%%%%%%%%%%%%%%%%%% TEACHING EXPERIENCE
%%%%%%%%%%%%%%%%%%%%%%%%%%

\section{Teaching Experience}

\href{http://www.tamu.edu/}{\textbf{Texas A\&M University}},
College Station, TX
\begin{outerlist}

\item[] \textit{Substitute Lecturer} \hfill \textbf{January 2016}
    \begin{innerlist}
        \item ECEN~322: Electric and Magnetic Fields
        \begin{innerlist}
            \item Undergraduate course
            \item Main instructor: Robert D. Nevels
        \end{innerlist}
    \end{innerlist}

\item[] \textit{Substitute Lecturer} \hfill \textbf{October 2015}
    \begin{innerlist}
        \item ECEN~445: Applied Electromagnetic Theory
        \begin{innerlist}
            \item Undergraduate course
            \item Main instructor: Robert D. Nevels
        \end{innerlist}
    \end{innerlist}

\item[] \textit{Substitute Lecturer} \hfill \textbf{January 2015}
    \begin{innerlist}
        \item ECEN~351: Applied Electromagnetics
        \begin{innerlist}
            \item Undergraduate course
            \item Main instructor: Robert D. Nevels
        \end{innerlist}
    \end{innerlist}

\end{outerlist}

\halfblankline

\href{http://www.uet.edu.pk/}{\textbf{University of Engineering \& Technology}},
KSK Campus, Pakistan
\begin{outerlist}

\item[] \textit{Lecturer}%
    \hfill \textbf{August 2009 to August 2012}
    \begin{innerlist}
        \item Instructor for EE~480: Antennas and Propagation
        \item Instructor for EE~380: Electromagnetic Theory
        \item Instructor for EE~381: Applied Electromagnetic Theory
    \end{innerlist}

\item[] \textit{Lab In-charge}
    \hfill \textbf{December 2009 to August 2012}\\
    \begin{innerlist}
        \item Set up Microwave and Antennas Laboratory
        \item Authored Antennas lab manual
        \end{innerlist}

        \item Lab Instructor for EE~360: Communication Systems
        \begin{innerlist}
            \item Spring 2012
        \end{innerlist}
\end{outerlist}

%%%%%%%%%%%%%%%%%%%%%%%%%%
%%%%%%%%%%%%%%%%%%%%%%%%%% AWARD
%%%%%%%%%%%%%%%%%%%%%%%%%%

\section{Honors}


\href{http://foreign.fulbrightonline.org/}{Fulbright Foreign Student}
\begin{innerlist}
    \item {Pursue Doctoral Degree at Texas A\&M University}, 2012--2017

\end{innerlist}

{Best Young Faculty}
\begin{innerlist}
    \item Department of Electrical Engineering \& Technology, UET Lahore, Pakistan, 2010-2011
\end{innerlist}

%%%%%%%%%%%%%%%%%%%%%%%%%%
%%%%%%%%%%%%%%%%%%%%%%%%%% SERVICE
%%%%%%%%%%%%%%%%%%%%%%%%%%

\section{Professional Service}

\textbf{Professional Memberships}
\begin{innerlist}
    \item IEEE Antennas \& Propagation Society
    \item IEEE Microwave Theory \& Techniques Society
    \item American Physical Society
\end{innerlist}

\halfblankline

\textbf{Referee Service}
\begin{innerlist}
    \item \emph{IEEE Antennas and Wireless Propagation Letters}
    \item \emph{IEEE Transactions on Antennas and Propagation}
    \item \emph{American Journal of Physics}
\end{innerlist}

\halfblankline


%%%%%%%%%%%%%%%%%%%%%%%%%%
%%%%%%%%%%%%%%%%%%%%%%%%%% SKILLS
%%%%%%%%%%%%%%%%%%%%%%%%%%

\section{Software and Hardware Skills}



Computer Programming:
%
\begin{innerlist}
    \item C, C$+$$+$
\end{innerlist}

\halfblankline

Numerical Analysis:
%
\begin{innerlist}
    \item \Matlab, Python
\end{innerlist}

\halfblankline

Desktop Editing and Productivity Software:
%
\begin{innerlist}
    \item Atom, Git
    \item \TeX{} (\LaTeX{}, \BibTeX{}, PSTricks),
    \item Microsoft Office, Google Docs
    \item TikZ, InkScape
\end{innerlist}

\halfblankline

Operating Systems:
%
\begin{innerlist}
    \item Microsoft Windows family, Linux (Ubuntu)
\end{innerlist}

\section{Expertise}

Mathematics:
%
\begin{innerlist}
    \item PDE, Stability Analysis, Linear Algebra, Fourier Transforms
\end{innerlist}

\halfblankline

Embedded and Real\-/time Systems:
%
\begin{innerlist}
    \item Software and hardware development with several MCU and
        DSP platforms (e.g., Atmel ATmega MCU's, Microchip PIC MCU's, Arduino and others)
\end{innerlist}

%%%%%%%%%%%%%%%%%%%%%%%%%%
%%%%%%%%%%%%%%%%%%%%%%%%%% ACADEMIC SERVICE
%%%%%%%%%%%%%%%%%%%%%%%%%%

\section{Student Mentoring}

\begin{bibsection}

    \item \textbf{Usman Samad}\\
        Undergraduate student in Electrical and Computer Engineering, Texas A\&M University.
        Modeling and Implementation of a Home Automation System
        2015.

\end{bibsection}




%%%%%%%%%%%%%%%%%%%%%%%%%%
%%%%%%%%%%%%%%%%%%%%%%%%%% REFERENCS
%%%%%%%%%%%%%%%%%%%%%%%%%%


\section{References Available to Contact}

Furnished upon request.
% \href
% {http://engineering.tamu.edu/electrical/people/rnevels}
% {\textbf{Dr.~Robert D.~Nevels}}
% (e\-/mail:~\href{mailto:nevels@ece.tamu.edu}{nevels@ece.tamu.edu}; phone:~+1-979-845-7591)
% %
% \begin{innerlist}
%     \item Professor,
%         \href{http://engineering.tamu.edu/electrical/}{Department of Electrical \& Computer Engineering},
%         \href{http://www.tamu.edu/}{Texas A\&M University}
%
%     \item[$\diamond$] 205K WEB, Texas A\&M University, College Station, TX
%         77843-3128
%
%     \item[$\star$] \emph{Dr.~Nevels is my doctoral supervisor.}
% \end{innerlist}
%
% \halfblankline
%
% \textbf{Dr.~Kai ~Chang}
% (e\-/mail:~\href{mailto:kaichang@tamu.edu}{kaichang@tamu.edu}; phone:~+1-979-845-5285)
% %
% \begin{innerlist}
%     \item Professor,
%         \href{http://engineering.tamu.edu/electrical/}{Department of Electrical \& Computer Engineering},
%         \href{http://www.tamu.edu/}{Texas A\&M University}
%
%     \item[$\diamond$] 205C WEB, Texas A\&M University, College Station, TX
%         77843-3128
%
%     \item[$\star$] \emph{Dr.~Chang has been my graduate courses teacher.}
% \end{innerlist}

% \halfblankline
%
%
%
% \halfblankline

% \nobibliography{my_citations}


\end{document}

%%%%%%%%%%%%%%%%%%%%%%%%%% End CV Document %%%%%%%%%%%%%%%%%%%%%%%%%%%%%

%----------------------------------------------------------------------%
% The following is copyright and licensing information for
% redistribution of this LaTeX source code; it also includes a liability
% statement. If this source code is not being redistributed to others,
% it may be omitted. It has no effect on the function of the above code.
%----------------------------------------------------------------------%
% Copyright (c) 2007, 2008, 2009, 2010, 2011 by Theodore P. Pavlic
%
% Unless otherwise expressly stated, this work is licensed under the
% Creative Commons Attribution-Noncommercial 3.0 United States License. To
% view a copy of this license, visit
% http://creativecommons.org/licenses/by-nc/3.0/us/ or send a letter to
% Creative Commons, 171 Second Street, Suite 300, San Francisco,
% California, 94105, USA.
%
% THE SOFTWARE IS PROVIDED "AS IS", WITHOUT WARRANTY OF ANY KIND, EXPRESS
% OR IMPLIED, INCLUDING BUT NOT LIMITED TO THE WARRANTIES OF
% MERCHANTABILITY, FITNESS FOR A PARTICULAR PURPOSE AND NONINFRINGEMENT.
% IN NO EVENT SHALL THE AUTHORS OR COPYRIGHT HOLDERS BE LIABLE FOR ANY
% CLAIM, DAMAGES OR OTHER LIABILITY, WHETHER IN AN ACTION OF CONTRACT,
% TORT OR OTHERWISE, ARISING FROM, OUT OF OR IN CONNECTION WITH THE
% SOFTWARE OR THE USE OR OTHER DEALINGS IN THE SOFTWARE.
%----------------------------------------------------------------------%
